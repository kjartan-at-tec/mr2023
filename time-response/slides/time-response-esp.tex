\documentclass[dvisvgm,hypertex,aspectratio=169]{beamer}
\usefonttheme{serif}

%\usepackage[draft]{animate}
\usepackage[final]{animate}
\usepackage{ifthen}

% \usepackage{khpreamble}


%%%%%%%%%%%%%%%%%%%%%%%%%%%%%%%%%%%%%%%%%%%%%%%%%%%%%%%%%%%%%%%%%%%%%%%%%%%%%%%
% PageDown, PageUp key event handling; navigation symbols
%%%%%%%%%%%%%%%%%%%%%%%%%%%%%%%%%%%%%%%%%%%%%%%%%%%%%%%%%%%%%%%%%%%%%%%%%%%%%%%
\usepackage[totpages]{zref}
\usepackage{atbegshi}
\usepackage{fontawesome}
\setbeamertemplate{navigation symbols}{}
\AtBeginShipout{%
  \AtBeginShipoutAddToBox{%
    \special{dvisvgm:raw
      <defs>
      <script type="text/javascript">
      <![CDATA[
        document.addEventListener('keydown', function(e){
          if(e.key=='PageDown'){
            \ifnum\thepage<\ztotpages
              document.location.replace('\jobname-\the\numexpr\thepage+1\relax.svg');%
            \fi
          }else if(e.key=='PageUp'){
            \ifnum\thepage>1
            %document.location.replace('\jobname-\the\numexpr\thepage-1\relax.svg');%
              document.location.replace('\jobname-\makeatletter\@anim@pad{2}{\thepage-1}\makeatother\relax.svg');%
            \fi%
          }
        });
      ]]>
      </script>
      </defs>
    }%
  }%
  \AtBeginShipoutUpperLeftForeground{%
    \raisebox{-\dimexpr\height+0.5ex\relax}[0pt][0pt]{\makebox[\paperwidth][r]{%
      \normalsize\color{structure!40!}%
      \ifnum\thepage>1%
      \href{\jobname-\the\numexpr\thepage-1\relax.svg}{\faArrowLeft}%
      \else%  
        \textcolor{lightgray}{\faArrowLeft}%  
      \fi\hspace{0.5ex}%
      \ifnum\thepage<\ztotpages%
      \href{\jobname-\the\numexpr\thepage+1\relax.svg}{\faArrowRight}%
      \else%
        \textcolor{lightgray}{\faArrowRight}%  
      \fi%
      \hspace{0.5ex}%
    }}%
  }%  
}%
%%%%%%%%%%%%%%%%%%%%%%%%%%%%%%%%%%%%%%%%%%%%%%%%%%%%%%%%%%%%%%%%%%%%%%%%%%%%%%%

\usepackage{tikz}
\usepackage{pgfplots}
\usepackage{pgfplotstable}
\pgfplotsset{compat=1.16}
\usetikzlibrary{calc}
\usepackage{amsmath}
\DeclareMathOperator{\sign}{sgn}


\author{Kjartan Halvorsen}
\date{2021-06-07}
\title{Respuesta de sistemas de primer y segunda orden}

% ------------------------------------------------
% Determine which slides to include
\includeonlyframes{%
PZ0,% Polos y ceros
T1,% First order systems
T20,% Second order systems
T21,% Second order systems
}
% ------------------------------------------------

%%%%%%%%%%%%%%%%%%%%%%%%%%%%%%%%%%%%%%%%%%%%%%%%%%%%%%%%%%%%%%%%%%%%%%%%%%%%%%%
% Define footer
\usepackage{ccicons}

\makeatletter
\setbeamertemplate{footline}
{
  \leavevmode%
  \hbox{%
  %\begin{beamercolorbox}[wd=.333333\paperwidth,ht=2.25ex,dp=1ex,center]{title in head/foot}%
    %\usebeamerfont{title in head/foot}\insertsubsection
  %\end{beamercolorbox}%
  %\begin{beamercolorbox}[wd=.333333\paperwidth,ht=2.25ex,dp=1ex,right]{date in head/foot}%
  %  \usebeamerfont{date in head/foot}\insertshortdate{}\hspace*{2em}
  %  \insertframenumber{} / \inserttotalframenumber\hspace*{2ex} 
  %\end{beamercolorbox}}%
  %\vskip0pt%
  \begin{beamercolorbox}[wd=.92\paperwidth,ht=2.25ex,dp=1ex,right]{author in head/foot}%
    \usebeamerfont{author in head/foot}\insertauthor
  \end{beamercolorbox}%
  \begin{beamercolorbox}[wd=.08\paperwidth,ht=2.25ex,dp=1ex,right]{date in head/foot}%
    \ccbysa
  \end{beamercolorbox}}%
  \vskip0pt%
}
\makeatother
%%%%%%%%%%%%%%%%%%%%%%%%%%%%%%%%%%%%%%%%%%%%%%%%%%%%%%%%%%%%%%%%%%%%%%%%%%%%%%%

\newcommand*{\laplace}[1]{\ensuremath{\mathcal{L}\big\{ #1 \big\}}}
\newcommand*{\mexp}[1]{\ensuremath{\mathrm{e}^{#1}}}

\begin{document}

\maketitle


\begin{frame}[label=PZ0]{Polos y ceros}

  \begin{align*}
    G(s) &= \frac{B(s)}{A(s)} = \frac{b_0s^m + b_1s^{m-1} + \cdots + b_m }{s^n + a_1s^{n-1} + \cdots + a_n } = \frac{ b_0\overbrace{(s-z_1)(s-z_2)\cdots (s-z_m)}^{\text{\(m\) \alert{ceros}}}}{ \underbrace{(s-p_1)(s-p_2)\cdots(s-p_n)}_{\text{\(n\) \alert{polos}}}}\\
    &= \frac{c_1}{s-p_1} + \frac{c_2}{s-p_2} + \cdots +\frac{c_n}{s-p_n} 
  \end{align*}
\end{frame}


\begin{frame}[label=T1]{Respuesta de escalón de sistemas de primer orden}

  \footnotesize
  
  \[\tau \frac{d}{dt} y + y = ku \quad \Leftrightarrow \quad Y(s) = \frac{y(0)}{\tau s + 1} + \frac{k}{\tau s + 1} U(s)\]
  \[ y(0) = 0, \qquad u(t) = \begin{cases} 1 & t \ge 0\\ 0 & t<0 \end{cases} \quad \overset{\mathcal{L}}{\longleftrightarrow}\quad  U(s) = \frac{1}{s} \]
  \[ y(t) = k\big( 1 - \mexp{-\frac{t}{\tau}} \big)\] 

  \def\snum{30}
  \def\nframes{40}
  \begin{center}
    \begin{animateinline}[controls, palindrome,]{8}
      \multiframe{\nframes}{n=0+1}{
        \begin{tikzpicture}[scale=0.7, transform shape]
          \pgfmathsetmacro{\aa}{-3.6 + \n*3.4/(\nframes-1)}
          \pgfmathsetmacro{\ttau}{-1/\aa}

          %\useasboundingbox (-4.5 cm, -2.5 cm) rectangle (11.5 cm, 4.5 cm);

          \draw[->, thin] (-4, 0) -- node[pos=0.9, below] {Re} (2, 0);
          \draw[->, thin] (0, -2) -- node[pos=0.9, left] {Im} (0, 2);

          \node[red] at (\aa, 0) {\Large $\times$};
          \node at (\aa, -0.5) {$-\frac{1}{\tau}$};
          
          \begin{axis}[%
            xshift = 3cm,
            yshift = -2cm,
            width=8cm,
            height=6cm,
            axis lines = middle,
            ytick={0.632, 1},
            yticklabels={$0.632 k$, $k$},
            xtick={\ttau},
            xticklabel={$\tau$},
            xlabel={$t$},
            ylabel={$y$},
            xmin=-0.5,
            xmax=6,
            ymin=0,
            ymax=1.3,
            ]
            \addplot [red!80!black, domain=0:6, samples=\snum, thick, smooth,] { 1-exp(\aa*x) };
            \addplot [red!80!black, thick] coordinates {(-1, 0) (0,0)};

            \draw[thin, dashed] (axis cs: \ttau, 0) -- (axis cs: \ttau, 0.632);
            \draw[thin, dashed] (axis cs:  0, 0.632) -- (axis cs: \ttau, 0.632);
          \end{axis}
        \end{tikzpicture}
      }
    \end{animateinline}
  \end{center}
\end{frame}
  
\begin{frame}[label=T20]{Respuesta de escalón de sistemas de segundo orden}

{  \tiny
  
  \[\frac{d^2}{dt^2} y + 2\zeta\omega_n \frac{d}{dt} y + \omega_n^2 y = ku \quad \Leftrightarrow \quad Y(s) = \frac{\dot{y}(0) + sy(0)}{s^2 + 2\zeta\omega_n + \omega_n^2} + \frac{k}{s^2 + 2\zeta\omega_n + \omega_n^2} U(s)\]
  \[ y(0) = 0,\, \dot{y}(0)=0, \qquad u(t) = \begin{cases} 1 & t \ge 0\\ 0 & t<0 \end{cases} \quad \overset{\mathcal{L}}{\longleftrightarrow}\quad  U(s) = \frac{1}{s} \]
  \[ \text{Sobreamortiguado (\(\zeta>1\)):}\; y(t) = k\big( 1 - \frac{a_2 \mexp{-t a_1} }{a_2-a_1}  + \frac{a_1 \mexp{-a_2t} }{\tau_1-\tau_2} \big)\]
  \[\text{Subamortiguado (\(0 < \zeta < 1\)):}\; y(t) = k\big( 1 - \frac{\mexp{-\zeta\omega_n t}}{\sqrt{1-\zeta^2}}\sin( \omega_n\sqrt{1-\zeta^2}t + \cos^{-1}(\zeta))\big)\] 
}
  \def\snum{60}
  \def\nframes{50}
  \begin{center}
    \begin{animateinline}[controls, palindrome,]{8}
      \multiframe{\nframes}{n=0+1}{
        \begin{tikzpicture}[scale=0.7, transform shape]
          \pgfmathsetmacro{\zz}{1.5 - \n*1.31/(\nframes-1)}
          \pgfmathtruncatemacro{\zzcheck}{100*\zz}

          \draw[->, thin] (-3.5, 0) -- node[pos=0.99, right] {\footnotesize Re} (2, 0);
          \draw[->, thin] (0, -2) -- node[pos=0.99, above] {\footnotesize Im} (0, 2);
          \draw[dashed, thin, black!60] (-1, 0) arc [start angle=180, end angle=90, radius=1];
          \draw[dashed, thin, black!60] (-1, 0) arc [start angle=180, end angle=270, radius=1];


          \ifnum\zzcheck>100
          
            \pgfmathsetmacro{\aaone}{-(\zz - sqrt(pow(\zz, 2) - 1))}
            \pgfmathsetmacro{\aatwo}{-(\zz + sqrt(pow(\zz, 2) - 1))}

            \pgfmathsetmacro{\ttauone}{-1/\aaone}
            \pgfmathsetmacro{\ttautwo}{-1/\aatwo}

            \pgfmathsetmacro{\ttauonevalue}{1 + \aatwo*exp(\aaone*\ttauone)/(\aaone-\aatwo) - \aaone*exp(\aatwo*\ttauone)/(\aaone-\aatwo)}

            \node[red] at (\aaone, 0) {\large $\times$};
            \node at (\aaone, -0.4) {$-a_1$};
            \node[red] at (\aatwo, 0) {\large $\times$};
            \node at (\aatwo, -0.4) {$-a_2$};
          
          \begin{axis}[%
            xshift = 4cm,
            yshift = -2cm,
            width=8cm,
            height=6cm,
            axis lines = middle,
            ytick=\empty,
            xtick=\empty,
            %xticklabel={$\tau$},
            xlabel={$t$},
            ylabel={$y$},
            xmin=-0.5,
            xmax=15,
            ymin=0,
            ymax=2.3,
            clip=false,
            ]
            \addplot [red!80!black, domain=0:15, samples=\snum, thick, smooth,] { 1 + \aatwo*exp(\aaone*x)/(\aaone-\aatwo) - \aaone*exp(\aatwo*x)/(\aaone-\aatwo) };
            \addplot [red!80!black, thick] coordinates {(-1, 0) (0,0)};
            \draw[thin, dashed] (axis cs:  \ttauone, \ttauonevalue) -- (axis cs: \ttauone, -0.1);
            \node[] at (axis cs: \ttauone, -0.2) {\tiny $\frac{1}{a_1}$};
            
          \end{axis}
          \else
            % Underdamped    
            \pgfmathsetmacro{\realpart}{-\zz}
            \pgfmathsetmacro{\impart}{sqrt(1-pow(\zz, 2))}

            \node[red] at (\realpart, \impart) {\large $\times$};
            \node at (\realpart, -0.4) {\tiny $-\zeta\omega_n$};
            \draw[thin] (0, \impart) -- (0.3, \impart);
            \draw[thin] (0, -\impart) -- (0.3, -\impart);
            \draw[thin] (\realpart, 0)  -- (\realpart, -0.3);
            \node[anchor=west] at (0.4, \impart) {\tiny $i\omega_n\sqrt{1-\zeta^2}$};
            \node[red] at (\realpart, -\impart) {\large $\times$};
            \node[anchor=west] at (0.4, -\impart) {\tiny $-i\omega_n\sqrt{1-\zeta^2}$};

            \pgfmathsetmacro{\tpeak}{3.1415/sqrt(1-pow(\zz,2))}
            \pgfmathsetmacro{\peakvalue}{1 - exp(\realpart*\tpeak)/sqrt(1-pow(\zz, 2))*sin(deg(\impart*\tpeak) + acos(\zz))}
          \begin{axis}[%
            xshift = 4cm,
            yshift = -2cm,
            width=8cm,
            height=6cm,
            axis lines = middle,
            ytick=\empty,
            xtick=\empty,
            %xticklabel={$\tau$},
            xlabel={$t$},
            ylabel={$y$},
            xmin=-0.5,
            xmax=15,
            ymin=0,
            ymax=2.3,
            clip = false,
            ]
            \addplot [red!80!black, domain=0:15, samples=\snum, thick, smooth,] { 1 - exp(\realpart*x)/sqrt(1-pow(\zz, 2))*sin(deg(\impart*x) + acos(\zz))};
            \addplot [red!80!black, thick] coordinates {(-1, 0) (0,0)};

            \ifnum\zzcheck<80
   
            \draw[thin, dashed] (axis cs:  \tpeak, \peakvalue) -- (axis cs: \tpeak, -0.1);
            \node[] at (axis cs: \tpeak, -0.2) {\tiny $t_p = \frac{\pi}{\omega_n\sqrt{1-\zeta^2}}$};
            \else
            \node[white] at (axis cs: 2, -0.2) {\tiny $t_p = \frac{\pi}{\omega_n\sqrt{1-\zeta^2}}$};

            \fi
          \end{axis}

          \fi
        \end{tikzpicture}
      }
    \end{animateinline}
  \end{center}
\end{frame}
  
\begin{frame}[label=T21]{Respuesta de escalón de sistemas de segundo orden}

{  \tiny
  
  \[\text{Subamortiguado (\(0 < \zeta < 1\)):}\; y(t) = k\big( 1 - \frac{\mexp{-\zeta\omega_n t}}{\sqrt{1-\zeta^2}}\sin( \omega_n\sqrt{1-\zeta^2}t + \cos^{-1}(\zeta))\big)\] 
}
  \def\snum{60}
  \def\nframes{20}
  \begin{center}
    \begin{animateinline}[controls, palindrome,]{8}
      \multiframe{\nframes}{n=0+1}{
        \begin{tikzpicture}[scale=0.7, transform shape]
          \pgfmathsetmacro{\zz}{0.4}
          \pgfmathsetmacro{\omegan}{0.3 + 0.1*\n}
          \pgfmathsetmacro{\realpartt}{-\zz*2}
          \pgfmathsetmacro{\impartt}{2*sqrt(1-pow(\zz, 2))}
          
          \draw[->, thin] (-3.5, 0) -- node[pos=0.99, right] {\footnotesize Re} (2, 0);
          \draw[->, thin] (0, -2) -- node[pos=0.99, above] {\footnotesize Im} (0, 2);
          \draw[dashed, thin, black!60] (0, 0) -- (\realpartt, \impartt);
          \draw[dashed, thin, black!60] (0, 0) -- (\realpartt, -\impartt);


            % Underdamped    
            \pgfmathsetmacro{\realpart}{-\zz*\omegan}
            \pgfmathsetmacro{\impart}{sqrt(1-pow(\zz, 2))*\omegan}

            \node[red] at (\realpart, \impart) {\large $\times$};
            \node at (\realpart, -0.4) {\tiny $-\zeta\omega_n$};
            \draw[thin] (0, \impart) -- (0.3, \impart);
            \draw[thin] (0, -\impart) -- (0.3, -\impart);
            \draw[thin] (\realpart, 0)  -- (\realpart, -0.3);
            \node[anchor=west] at (0.4, \impart) {\tiny $i\omega_n\sqrt{1-\zeta^2}$};
            \node[red] at (\realpart, -\impart) {\large $\times$};
            \node[anchor=west] at (0.4, -\impart) {\tiny $-i\omega_n\sqrt{1-\zeta^2}$};

            \pgfmathsetmacro{\tpeak}{3.1415/sqrt(1-pow(\zz,2))/\omegan}
            \pgfmathsetmacro{\peakvalue}{1 - exp(\realpart*\tpeak)/sqrt(1-pow(\zz, 2))*sin(deg(\impart*\tpeak) + acos(\zz))}
          \begin{axis}[%
            xshift = 4cm,
            yshift = -2cm,
            width=8cm,
            height=6cm,
            axis lines = middle,
            ytick=\empty,
            xtick=\empty,
            %xticklabel={$\tau$},
            xlabel={$t$},
            ylabel={$y$},
            xmin=-0.5,
            xmax=15,
            ymin=0,
            ymax=2.3,
            clip = false,
            ]
            \addplot [red!80!black, domain=0:15, samples=\snum, thick, smooth,] { 1 - exp(\realpart*x)/sqrt(1-pow(\zz, 2))*sin(deg(\impart*x) + acos(\zz))};
            \addplot [red!80!black, thick] coordinates {(-1, 0) (0,0)};

            \draw[thin, dashed] (axis cs:  \tpeak, \peakvalue) -- (axis cs: \tpeak, -0.1);
            \node[] at (axis cs: \tpeak, -0.2) {\tiny $t_p = \frac{\pi}{\omega_n\sqrt{1-\zeta^2}}$};
          \end{axis}

        \end{tikzpicture}
      }
    \end{animateinline}
  \end{center}
\end{frame}
  
\end{document}
