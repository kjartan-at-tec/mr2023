\documentclass[]{standalone}
\usepackage{graphicx}


\usepackage{tikz}
\usepackage{pgfplots}
\usepackage{pgfplotstable}
\pgfplotsset{compat=1.16}
\usetikzlibrary{calc}
\usetikzlibrary{decorations.pathmorphing,patterns,decorations.markings}

% Horizontal dashpot with casing fixed to origin and extending to the right 
% Args
% 1: dashpot length. Should be the same throughout animation. Resting length is half
% 2: total length
\newcommand{\dashpot}[2]{%
  \pgfmathsetmacro{\restlength}{1.3*(#1)}
  \pgfmathsetmacro{\rodlength}{0.65*\restlength}
  \pgfmathsetmacro{\dplhalf}{#1 / 2}
  \pgfmathsetmacro{\dpattl}{#1 / 6}
  \pgfmathsetmacro{\dwidth}{1} % in mm
  \pgfmathsetmacro{\dinnerwidth}{0.8} % in mm

  % Casing
  \draw (0, 0) -- (\dpattl, 0) -- ++(0, \dwidth mm) -- ++ (#1, 0);
  \draw (\dpattl, 0) -- ++(0, -\dwidth mm) -- ++ (#1, 0);
  % Rod
  \draw (#2, 0) -- ++ (-\rodlength, 0) -- ++(0, \dinnerwidth mm);
  \draw (#2, 0)  ++ (-\rodlength, 0) -- ++(0, -\dinnerwidth mm);
}

\begin{document}

\def\masswidth{2}
\def\massheight{1.6}
\def\springlen{5}

\begin{tikzpicture}[scale=0.8]
  \tikzstyle{ground}=[fill,pattern=north east lines,draw=none,minimum width=0.75cm,minimum height=0.3cm]
  \tikzstyle{damper}=[decoration={markings,  
    mark connection node=dmp,
  mark=at position 0.5 with 
  {
    \node (dmp) [thick,inner sep=0pt,transform shape,rotate=-90,minimum width=15pt,minimum height=3pt,draw=none] {};
    \draw [] ($(dmp.north east)+(2pt,0)$) -- (dmp.south east) -- (dmp.south west) -- ($(dmp.north west)+(2pt,0)$);
    \draw [] ($(dmp.north)+(0,-5pt)$) -- ($(dmp.north)+(0,5pt)$);
  }
}, decorate]
\tikzstyle{spring}=[decorate,decoration={zigzag,pre length=0.2cm,post length=0.2cm,segment length=6}]

  \node (M1) [thick, draw, minimum width=\masswidth cm, minimum height=\massheight cm] {$m_1$};
  \node (M2) [thick, draw, minimum width=\masswidth cm, minimum height=\massheight cm, right of=M1,
  node distance=5cm] {$m_2$};

  \node (ground) [ground,anchor=north west, xshift = -1.5cm, yshift=-0.25cm,minimum width=8cm] at (M1.south) {};
  \draw (ground.north east) -- (ground.north west);

  \node (wall) [ground, rotate=-90, minimum width=3cm,yshift=-3cm] {};
  \draw (wall.north east) -- (wall.north west);

  \node (wall2) [ground, rotate=-90, minimum width=3cm,yshift=8cm] {};
  \draw (wall2.south east) -- (wall2.south west);

  \draw[spring] (wall.north)  --  node[below] {$k_1$} (M1.west);
  \draw[damper] (wall2.south)  --  node[below, pos=0.3] {$b$} (M2.east);

  \coordinate (m1s) at ($(M1.east) + (0,0.5cm)$);
  \coordinate (m2s) at ($(M2.west) + (0,0.5cm)$);
  \coordinate (m1d) at ($(M1.east) + (0,-0.5cm)$);
  \coordinate (m2d) at ($(M2.west) + (0,-0.5cm)$);

  \draw[spring] (m1s)  --  node[above] {$k_2$} (m2s);
  \draw[damper] (m1d)  --  node[below, pos=0.3] {$b$} (m2d);

  \draw [] (M1.south west) ++ (0.2cm,-0.125cm) circle (0.125cm)  (M1.south east) ++ (-0.2cm,-0.125cm) circle (0.125cm);
  \draw [] (M2.south west) ++ (0.2cm,-0.125cm) circle (0.125cm)  (M2.south east) ++ (-0.2cm,-0.125cm) circle (0.125cm);

  \draw[<-, thick] (M2.north east) ++(0,-0.2cm) -- ++(1cm, 0) node[right] {$F(t)$};

  \draw[dashed] (M1.north west) -- ++(0, 14mm);
  \draw[dashed] (M2.north west) -- ++(0, 14mm);

  \draw[->] (M1.north west) ++ (0, 8mm) -- ++(1cm, 0) node[right] {$z_1$};
  \draw[->] (M2.north west) ++ (0, 8mm) -- ++(1cm, 0) node[right] {$z_2$};
  

\end{tikzpicture}
\end{document}
