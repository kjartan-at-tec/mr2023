\documentclass[dvisvgm,hypertex,aspectratio=169]{beamer}
\usefonttheme{serif}

%\usepackage[utf8]{inputenc}
%\usepackage[T1]{fontenc}

%\usepackage[draft]{animate}
\usepackage[final]{animate}
\usepackage{ifthen}


\usepackage{pythontex} % <--
\usepackage{graphicx}


\usepackage{tikz}
\usepackage{pgfplots}
\usepackage{pgfplotstable}
\pgfplotsset{compat=1.16}
\usetikzlibrary{calc}
\usetikzlibrary{decorations.pathmorphing,patterns}
\usepackage{amsmath}


%%%%%%%%%%%%%%%%%%%%%%%%%%%%%%%%%%%%%%%%%%%%%%%%%%%%%%%%%%%%%%%%%%%%%%%%%%%%%%%
% PageDown, PageUp key event handling; navigation symbols
%%%%%%%%%%%%%%%%%%%%%%%%%%%%%%%%%%%%%%%%%%%%%%%%%%%%%%%%%%%%%%%%%%%%%%%%%%%%%%%
\usepackage[totpages]{zref}
\usepackage{atbegshi}
\usepackage{fontawesome}
\setbeamertemplate{navigation symbols}{}
\AtBeginShipout{%
  \AtBeginShipoutAddToBox{%
    \special{dvisvgm:raw
      <defs>
      <script type="text/javascript">
      <![CDATA[
        document.addEventListener('keydown', function(e){
          if(e.key=='PageDown'){
            \ifnum\thepage<\ztotpages
              document.location.replace('\jobname-\the\numexpr\thepage+1\relax.svg');%
            \fi
          }else if(e.key=='PageUp'){
            \ifnum\thepage>1
            %document.location.replace('\jobname-\the\numexpr\thepage-1\relax.svg');%
              document.location.replace('\jobname-\makeatletter\@anim@pad{2}{\thepage-1}\makeatother\relax.svg');%
            \fi%
          }
        });
      ]]>
      </script>
      </defs>
    }%
  }%
  \AtBeginShipoutUpperLeftForeground{%
    \raisebox{-\dimexpr\height+0.5ex\relax}[0pt][0pt]{\makebox[\paperwidth][r]{%
      \normalsize\color{structure!40!}%
      \ifnum\thepage>1%
      \href{\jobname-\the\numexpr\thepage-1\relax.svg}{\faArrowLeft}%
      \else%  
        \textcolor{lightgray}{\faArrowLeft}%  
      \fi\hspace{0.5ex}%
      \ifnum\thepage<\ztotpages%
      \href{\jobname-\the\numexpr\thepage+1\relax.svg}{\faArrowRight}%
      \else%
        \textcolor{lightgray}{\faArrowRight}%  
      \fi%
      \hspace{0.5ex}%
    }}%
  }%  
}%
%%%%%%%%%%%%%%%%%%%%%%%%%%%%%%%%%%%%%%%%%%%%%%%%%%%%%%%%%%%%%%%%%%%%%%%%%%%%%%%

%%%%%%%%%%%%%%%%%%%%%%%%%%%%%%%%%%%%%%%%%%%%%%%%%%%%%%%%%%%%%%%%%%%%%%%%%%%%%%% 
% Define footer
\usepackage{ccicons}

\makeatletter
\setbeamertemplate{footline}
{
  \leavevmode%
  \hbox{%
  %\begin{beamercolorbox}[wd=.333333\paperwidth,ht=2.25ex,dp=1ex,center]{title in head/foot}%
    %\usebeamerfont{title in head/foot}\insertsubsection
  %\end{beamercolorbox}%
  %\begin{beamercolorbox}[wd=.333333\paperwidth,ht=2.25ex,dp=1ex,right]{date in head/foot}%
  %  \usebeamerfont{date in head/foot}\insertshortdate{}\hspace*{2em}
  %  \insertframenumber{} / \inserttotalframenumber\hspace*{2ex} 
  %\end{beamercolorbox}}%
  %\vskip0pt%
  \begin{beamercolorbox}[wd=.92\paperwidth,ht=2.25ex,dp=1ex,right]{author in head/foot}%
    \usebeamerfont{author in head/foot}\insertauthor
  \end{beamercolorbox}%
  \begin{beamercolorbox}[wd=.08\paperwidth,ht=2.25ex,dp=1ex,right]{date in head/foot}%
    \ccbysa
  \end{beamercolorbox}}%
  \vskip0pt%
}
\makeatother
%%%%%%%%%%%%%%%%%%%%%%%%%%%%%%%%%%%%%%%%%%%%%%%%%%%%%%%%%%%%%%%%%%%%%%%%%%%%%%%


\author{\href{mailto:kjartan@tec.mx}{kjartan@tec.mx}}


% Vertical dashpot with casing fixed to origin and extending downwards 
% Args
% 1: dashpot length. Should be the same throughout animation. Resting length is half
% 2: total length
\newcommand{\dashpot}[2]{%
  \pgfmathsetmacro{\restlength}{1.3*(#1)}
  \pgfmathsetmacro{\rodlength}{0.65*\restlength}
  \pgfmathsetmacro{\dplhalf}{#1 / 2}
  \pgfmathsetmacro{\dpattl}{#1 / 6}
  \pgfmathsetmacro{\dwidth}{1} % in mm
  \pgfmathsetmacro{\dinnerwidth}{0.8} % in mm

  % Casing
  \draw (0, 0) -- (0, -\dpattl) -- ++(\dwidth mm, 0) -- ++ (0, -#1);
  \draw (0, -\dpattl) -- ++(-\dwidth mm, 0) -- ++ (0, -#1);
  % Rod
  \draw (0, -#2) -- ++ (0, \rodlength) -- ++(\dinnerwidth mm, 0);
  \draw (0, -#2)  ++ (0, \rodlength) -- ++(-\dinnerwidth mm, 0);
}

\begin{document}

\section{Simulate the model}


\section{Create animation}

\IfFileExists{./quarter_model.dta}{%
\def\wheeldiam{3}
\def\susplen{5}
\def\massheight{2}
\def\springelementsone{40}
\def\springelementstwo{32}
\pgfmathsetmacro{\dponelen}{(\susplen-\massheight)/1.8}
\pgfplotstableread[col sep=comma]{quarter_model.dta}{\qtable}
\begin{frame}[label=Q]{Quarter model}
  \begin{center}
    \begin{animateinline}[controls, loop,]{20}
      \multiframe{120}{n=0+1}{
        \pgfplotstablegetelem{\n}{1}\of\qtable
        \pgfmathsetmacro{\yone}{\wheeldiam + \susplen + \pgfplotsretval}
        \pgfplotstablegetelem{\n}{2}\of\qtable
        \pgfmathsetmacro{\ytwo}{\wheeldiam + \pgfplotsretval}
        \pgfplotstablegetelem{\n}{3}\of\qtable
        \pgfmathsetmacro{\yroad}{0.6*\pgfplotsretval}
        \pgfmathsetmacro{\xroad}{-\n*6/18}
        \pgfmathsetmacro{\displone}{abs(\yone-\ytwo)-\massheight}
        \pgfmathsetmacro{\springone}{abs(\yone-\ytwo)/\springelementsone}
        \pgfmathsetmacro{\springtwo}{abs(\ytwo-\yroad)/\springelementstwo}
        \begin{tikzpicture}[scale=0.5, transform shape]
          \huge
          \clip (-5, -2) rectangle (6.5, 11);
          \begin{scope}[xshift=\xroad cm,]
            \draw[thick] (-6 , -0.35) -- (6.6 ,-0.35) arc [start angle=130, end angle=50, radius = 1.8 cm] -- (20, -0.35) arc [start angle=230, end angle=310, radius=3.6cm] -- (60, -0.35);
          \end{scope}
          \node[circle, inner sep=2mm, draw] (wheel) at (0, \yroad) {};
          \node[draw, red!80!black, thick, minimum height=\massheight cm, minimum width=4cm] (m1) at (0, \yone) {$m_1$};
          \node[draw, orange!80!black, thick, minimum height=\massheight cm, minimum width=2cm] (m2) at (0, \ytwo) {$m_2$};
          \draw[decoration={aspect=0.3, segment length=\springone cm, amplitude=1.5mm,coil},decorate] (m1.south) ++ (-0.4, 0) -- ++(0, -\displone);
          \begin{scope}[shift={($(m1.south)+(0.4,0)$)}]
            \dashpot{\dponelen}{\displone}
          \end{scope}
          \draw[decoration={aspect=0.3, segment length=\springtwo cm, amplitude=1.5mm,coil},decorate] (m2.south) -- (wheel); 

        \end{tikzpicture}
      }
    \end{animateinline}
  \end{center}
\end{frame}
}{}
\end{document}


